\documentclass{article}
\usepackage{graphicx} % Required for inserting images

\title{University Course Assignment Problem}
\author{
  Mahir Jimit Ghadiali \\
  Jainam Shah \\
  Michael Lewis
}
\date{November 2023}

\begin{document}

\maketitle

\section{Introduction}
The focal point of our research revolves around optimizing the University Course Assignment System. Within a department, faculty members are classified into three distinct groups: ``x1,'' ``x2,'' and ``x3.'' Each group has a designated course load, with ``x1'' handling 0.5 courses per semester, ``x2'' managing 1 course per semester, and ``x3'' overseeing 1.5 courses per semester.

This system allows faculty members the flexibility to enroll in multiple courses during a semester, and conversely, a single course may be assigned to multiple faculty members. When a course is shared between two professors, each professor's load is considered to be 0.5 courses. Additionally, each faculty member maintains a preference list of courses, ordered by personal preference, with the most favored courses listed at the top. It is important to note that there is no prioritization among faculty members within the same category.

The primary objective of this research problem is to devise an assignment scheme that maximizes the number of courses assigned to faculty while adhering to their preferences and the category-based constraints (``x1,'' ``x2,'' ``x3''). The challenge lies in ensuring that a course can only be assigned to a faculty member if it is present in their preference list.

This problem stands out due to the flexibility it offers regarding the number of courses faculty members can undertake, setting it apart from typical Assignment problems. Possible modifications may involve adjusting the maximum number of courses ``y'' for each category of professors, rather than requiring strict adherence, or expanding the number of professor categories beyond the existing three to create a more generalized solution.

\end{document}
